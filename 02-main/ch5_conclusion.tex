\chapter{Conclusion}
\label{ch:conclusion}

The repository containing all the code sources of our project is available at the following link:


\centerline{\url{https://gitlab.com/Skogarmadr/mse-pa}}
~\\
The repository containing the code source of this report is available at the following link:


\centerline{\url{https://github.com/crypt0log-hai/mse-pa-report}}


~\\\\


This project has successfully met its objectives of exploring the potential of blockchain technology to mitigate the current
payment systems limitations today by creating our own cryptocurrency for international settlements. We have developed two proofs of concept, including a token wallet and a secure exchange platform to trade tokens. Our new cryptocurrency is built using the ERC20 standard, and it incorporates critical features which allow us to create a robust and secure token smart contract providing fungibility and interoperability.


Our token wallet allows users to securely store and manage our and other ERC20 tokens, providing
a convenient and user-friendly interface. By implementing features such as balance checking, token transfer,
and token approval, we have created a reliable and efficient solution for managing our cryptocurrency and other cryptocurrencies
using the Ethereum blockchain.


Through the development of a secure exchange platform, we allow users to trade their tokens with other users' tokens. Furthermore, implementing the ERC20 standard enables users to buy and sell token pairs at a determined price through
an order book. Therefore, we have created a decentralized exchange ensuring fair and transparent trading among users,
which valid the need to eliminate the trusted third party, such as a bank or a broker, enhancing transparency.


Finally, we have successfully achieved the objectives of this project by creating a new token ecosystem that meets the requirements
for international settlements. Furthermore, we have demonstrated that our virtual currency complying with the ERC20 standard, can be traded to settle in a secure and transparent exchange platform between two parties.


There are still opportunities for future work to improve the platform exchange and the tokens on our project for enhancements and
new features. One alternative is exploring other token standards, such as ERC777, to mitigate the risk of misusing
the ERC20 standard and provide some additional features. Furthermore, we can also explore implementing a liquidity mechanism
and expanding the platform exchange to support more tokens and token pairs.


Overall, the objectives for this project have been met, giving a solid insight into the development and implementation
of a blockchain-based exchange platform. We are confident that our solution can be viewed as a foundation for
future work in blockchain technology and maybe even be integrated into the financial sector.
