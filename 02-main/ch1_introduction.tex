\chapter{Introduction}
\label{ch:introduction}

This section aims to provide an overview of the context of the project, the state of the art, and the project's objectives. It introduces the
problem of current payment systems and the potential of blockchain technology to address some of the limitations of these systems. It also presents the
objectives of the project and the methodology that will be used to achieve them.


\minitoc

% \newpage

% -----------------------------------------------------------------------------

\section{Context}
\label{sec:ch1_context}

The exchange of goods and services has been a part of the beginning of human society throughout our history. It results in the creation of various payment methods
to facilitate transactions. From the barter system in the early days to the introduction of currency in the roman empire, humans
have always found ways for making transactions with each other in a fair trade. 
Nowadays, the most common payment methods are through bank systems, which are based on the use of fiat currencies. However these traditional payment systems
suffer high transaction costs, lengthy settlement times, and, most important, transparency.

High transaction costs associated with traditional systems, such as fees charged by financial institutions for international transactions,
impose a financial burden on businesses and individuals. Moreover, the lengthy settlement times for
transactions, especially international ones, create delays and hinder the smooth flow of funds in a globalized world.

The challenge is more transparency in traditional payment systems, impacting the trust between parties. Indeed,
multiple intermediaries or third parties can be involved in financial transactions, rendering the transactions unclear. Moreover, this opacity opens
the door to fraud, errors, and illegal activities, posing risks to businesses and individuals.


While alternative payment systems like the Hawala system have emerged in certain regions, they also have limitations. Based on trust and personal relationships, the Hawala system offers low-cost and faster transactions, particularly in areas lacking access to formal banking services. However, it lacks scalability and regulatory systems, raising concerns about transparency and potential risks associated with fraud and illegal activities.


Overall, traditional payment systems have cost, settlement times, and transparency limitations. Cryptocurrencies and blockchain technology offer a
a viable alternative that addresses these challenges.


\section{State of the art}
\label{sec:state_of_the_art}

% Introduce the concept of blockchain and cryptocurrencies

This chapter provides an overview of blockchain technology's state-of-the-art and critical features. It also addresses the challenges 
associated with international settlements, as discussed in Section \ref{sec:ch1_context}, and explores how blockchain technology, specifically 
the Ethereum blockchain, can offer potential solutions. The focus is on developing a new cryptocurrency token shaped for international settlements, 
taking advantage of the capabilities and features provided by the Ethereum blockchain. 

\subsubsection{Blockchain technology and its features}

Blockchain technology has become essential for designing and developing secure exchange systems between multiple parties. 
As the growth of multilateral trade and globalization increases, the need for a secure and transparent payment 
system is becoming more critical to settling currencies. Blockchain-based payment systems can provide a solution where virtual currencies 
(cryptocurrencies) can settle transactions between pairs of national currencies. The first cryptocurrency was Bitcoin, created in 
2008 \cite{online_satoshinakamoto} by an unknown person or group under the pseudonym Satoshi Nakamoto. We can name Bitcoin
, the first decentralized cryptocurrency introduced the concept of a decentralized and distributed system for 
financial transactions. After Bitcoin, other cryptocurrencies have been created since then, such as Ethereum, Monero, Litecoin, etc.

The blockchain acts like a digital ledger that enables the secure and transparent transfer of digital assets
between two parties without needing a trusted intermediary. The immutability of the blockchain is achieved through
cryptography and consensus algorithms, providing high security and transparency. References such as \textit{Blockchain Basics} by
Daniel Dresched \cite{Drescher2017-hj} and \textit{Mastering Bitcoin} by Andreas Antonopoulos \cite{Antonopoulos2017-et} provide an excellent introduction
to blockchain technology.

\subsubsection{Current international settlements and their challenges}
Traditional international settlements must be more robust to limitations such as high transaction costs, lengthy settlement times
, and lack of transparency. These limitations are due to the reliance on trusted intermediaries and third parties to process
and verify transactions. In his book \textit{Global Financial Systems} \cite{Danielsson2013-cw}, Jon Danielsson offers
valuable insights into the current challenges of international settlements.

\subsubsection{Blockchain technology versus these challenges}

Blockchain technology can offer some promising solutions to the current challenges of international settlements. Due to its decentralized and distributed nature, it can provide
a low-cost, fast, and secure payment system. Settlements can be executed directly
between two parties reducing the need for intermediaries and third parties and thus reducing the transaction costs. The immutability and
transparency of the blockchain can also provide real-time visibility of transactions, reducing the risk of fraud and illegal activities.
In the book of \textit{Blockchain Revolution} by Don Tapscott and Alex Tapscott \cite{Tapscott2018-ed}, the authors explore the potential of blockchain
technology to transform the financial sector and the economy.

\subsubsection{Ethereum and smart contracts}
The Ethereum blockchain is the second most popular blockchain after Bitcoin among developers with its new features allowing to develop smart contracts and the concept of
Decentralized Applications (dApps). A smart contract is a digital one stored on the blockchain and executed automatically when certain conditions are met, ensuring automated
and transparent settlement transactions. The book of \textit{Mastering Ethereum} by Andreas Antonopoulos and Gavin Wood \cite{Antonopoulos2018-wp} provides
in-depth insights into the potential of Ethereum blockchain and Smart Contracts development.


\section{Objectives of this project}

Our objective aims to develop a new cryptocurrency token for international settlements using the Ethereum blockchain. The token will be used to settle transactions
between two or multiple parties. For this project, to have a real case, we will consider the case of a company that wants to use its cryptocurrency token to settle transactions with its suppliers, customers, or employees.
To project the scenario further, we will consider the case of a company that can exchange its cryptocurrency token with other cryptocurrencies or national currencies (e.g. USD) on a secure Blockchain-based inter-currencies settlement
platform.






